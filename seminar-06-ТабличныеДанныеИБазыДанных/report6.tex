\documentclass{ou}
\begin{document}
\thispagestyle{empty}
\begin{center}
\minobrRF\\
\FGAVO\\
\textbf{\sixteen{\university}}\\
\institute\\
\department\\[1\baselineskip]
{\fontsize{16}{24}\selectfont\bfseries \work}\\[0.5\baselineskip]
\discipline\\
Технологии создания и обработки табличных данных\\[1\baselineskip]
\end{center}
\instructor \hfill \selectfont \Nigmatullin\\[1\baselineskip]
\students \hfill \selectfont \group\\
\vfill
\begin{center}
    \city
\end{center}
\newpage
\tableofcontents
\newpage
\section{Общие принципы организации и хранения данных в базах данных}
База данных (БД) – совокупность данных, организованных по определённым правилам, предусматривающим общие принципы описания, хранения и манипулирования данными, независимая от прикладных программ. Эти данные относятся к определённой предметной области и организованы таким образом, что могут быть использованы для решения многих задач многими пользователями. 

Система управления базами данных (СУБД) – это специальный вид программного обеспечения, который позволяет управлять и организовывать данные. В отличие от простого хранения данных в файловой системе, СУБД предоставляет удобные механизмы для поиска, обновления, добавления и удаления данных. 

Базы данных нужны для хранения больших объёмов структурированных данных, где важны безопасность, надёжность и возможность интеграции с другими системами. Хранение данных в файлах имеет недостатки, которые делают их менее эффективными для сложных задач.

\subsection{Организация данных}
К организации данных в системах автоматизированной обработки информации возможны два подхода: 

1) Каждый пользователь системы создает наборы данных, необходимых для решения его задач, и пишет программы обработки данных. Например, в рамках ВУЗа различные подразделения (деканат, отдел кадров, бухгалтерия и т.п.) могут создать свои подсистемы, предназначенные для решения определенных задач;

2) Вся информация, описывающая определенную предметную область, хранится, интегрировано, в единой базе данных (БД) и каждый из пользователей имеет доступ к тем данным, которые необходимы ему для решения его задач;

Первый из подходов имеет ряд недостатков:

- В различных подсистемах часто хранится одна и та же информация (сведения о студентах, о преподавателях и т.п.), т.е. данные дублируются, и возникает избыточность информации. При появлении изменений в данных необходимо обновлять многочисленные наборы данных и, если отдельные наборы окажутся не скорректированы, возникнет противоречивость данных;

- Обмен данными между отдельными подсистемами затруднен или невозможен, т.к. прикладные программы отдельных подсистем написаны на различных языках программирования, а данные представлены в различных форматах;

- При появлении в подсистеме новых задач, а, следовательно, и новых данных придется вносить изменения в уже созданные файлы и программы, т.к. данные описаны в каждой из прикладных программ (описаны типы и форматы данных, типы файлов). В подобном случае говорят, что прикладные программы зависят от хранимых данных;

Существенным достоинством первого подхода является наличие у каждого набора данных единственного владельца, что снижает риск неавторизованного доступа к данным, их искажения и разрушения.

При хранении данных в БД перечисленные недостатки снимаются. Однако в этом случае возникает другой недостаток: у данных нет единого хозяина. Из-за этого снижается ответственность за правильность хранимых данных и нарушается секретность. Для устранения этого недостатка для БД разрабатывается специальная система защиты.

Один из основных факторов построения базы данных заключается в том, что она должна соответствовать специфическим требованиям конкретной задачи или приложения. Для этого необходимо провести анализ данных, определить структуру и свойства объектов, которые будут храниться в базе данных.

Принцип построения базы данных состоит в том, чтобы организовать данные в такой форме, чтобы они были легко доступны и могли быть обработаны с помощью компьютера. База данных представляет собой структурированную коллекцию данных, которая хранится в компьютере и доступна для использования.

Для построения базы данных необходимо определить ее цели и требования к данным. Например, если база данных предназначена для хранения информации о клиентах, то необходимо определить, какие данные о клиентах будут храниться, как они будут структурированы и каким образом они будут связаны между собой.

При проектировании базы данных необходимо учитывать следующие принципы:

1) Нормализация данных. Данные должны быть разбиты на отдельные таблицы и связаны между собой отношениями. Это позволяет уменьшить дублирование данных и обеспечить их целостность;

2) Соответствие типов данных. Каждый столбец в таблице должен иметь соответствующий тип данных, чтобы избежать ошибок при работе с данными;

3) Определение первичного ключа. Каждая таблица должна иметь уникальный идентификатор, который позволяет однозначно идентифицировать каждую запись в таблице;

4) Установление связей между таблицами. Данные в разных таблицах должны быть связаны между собой отношениями, чтобы обеспечить целостность данных и возможность их последующей обработки;

5) Определение индексов. Индексы позволяют ускорить поиск данных в таблицах и улучшить производительность базы данных;

6) Определение ограничений на данные. Ограничения позволяют контролировать ввод данных и обеспечить их правильность и целостность.

В основе построения БД лежат определенные научные принципы, позволяющие создавать высококачественные системы, отвечающие современным требованиям. Из множества используемых принципов создания БД выделяются наиболее существенные:

- интеграции данных;

- централизации управления данными.

Оба принципа отражают суть БД. Интеграция является основой организации БД, централизация управления – основой организации и функционирования СУБД. 

Суть принципа интеграции данных состоит в объединении отдельных, взаимно не связанных данных в единое целое, в роли которого выступает база данных, в результате чего пользователю и его прикладным программам все данные представляются единым информационным массивом. Следование принципу интеграции обеспечивает:

- упрощение поиска взаимосвязанных данных и их совместную обработку;

- уменьшение избыточности данных;

- упрощение процесса ведения БД.

Принцип централизации управления состоит в передаче всех функций управления данными единому комплексу управляющих программ – СУБД.

\subsection{Требования, предъявляемые к БД}
Данные в БД не должны дублироваться. Избыточность данных, если она существует, влечет две опасности: 

– неоправданно большой расход памяти и уменьшение времени отклика системы при обработке излишне больших объемов данных;

– нарушение непротиворечивости данных, т.е. возникновение такой ситуации, когда в различных местах машинной памяти хранятся противоречивые данные. Возникновение противоречивости чрезвычайно опасно для БД.

Противоречивость может возникнуть в результате корректировки избыточных данных. При внесении изменений в логическую запись может случиться так, что отдельные экземпляры этой записи, хранящиеся в различных местах машинной памяти, окажутся нескорректированы. Противоречивость может возникнуть и при корректировке не избыточных данных.

В БД должны храниться только правильные данные, т.е. соблюдаются логические условия, в соответствии с которыми данные считаются правильными. Разрушение и искажение данных возможно в результате неосторожных действий пользователей, в результате ошибок в программах и сбоев оборудования. 

Для обеспечения целостности на данные, хранящиеся в БД, накладывают ограничения. При этом определяются условия, которым должны соответствовать значения данных. Например, один и тот же служащий не может иметь два различных года рождения и т.п.. Подобные ограничения называются законами БД. Выполнимость законов БД периодически проверяется СУБД. 

Для предотвращения возможности ввода неправильных данных разрабатываются средства контроля правильности вводимых данных. Например, можно использовать процедуры, проверяющие принадлежность вводимых значений определенному диапазону допустимых значений. Например, количество рабочих дней ограничивается сверху количеством дней в текущем месяце.

Целостность данных может нарушиться при неудачном завершении транзакции. Транзакцией называется некоторая неделимая последовательность операций над данными, выполняемая по одному запросу к БД. Примером транзакции является операция перевода денег с одного счета на другой в банковской системе. Здесь необходимо последовательное выполнение нескольких операций. Деньги снимаются с одного счета, данные корректируются, затем деньги добавляются к другому счету и данные вновь корректируются. Если хотя бы одно из действий не выполняется успешно, результат транзакции окажется неверным. СУБД должна отслеживать ход выполнения транзакции от начала до ее завершения. Если по какой-то причине какая-либо из операций не выполнилась, то транзакция отменяется полностью. При этом выполняется «откат» путем отмены всех уже выполненных изменений.

В БД должны быть предусмотрены средства восстановления данных после программных сбоев и сбоев оборудования. Существуют программы создания резервных копий и специальные программы, которые автоматически фиксируют любые внесенные в БД изменения (создается файл корректур). Если текущая версии БД испорчена, то берется предыдущая версия, в нее вносятся изменения, зафиксированные в файле корректур, и текущее (актуальное) состояние БД восстанавливается. 

Различные СУБД в той или иной мере располагают средствами обеспечения целостности данных. В противном случае такие средства разрабатываются системным программистом.

Прикладные программы не должны зависеть от хранимых данных, т.е. от способа хранения данных в физической памяти. Это позволяет добавлять в БД новые данные, изменять структуры хранения данных, создавать на БД новые приложения. Ранее созданные программы при этом не должны «чувствовать» эти изменения.  

Структура БД должна позволять включать новые и удалять устаревшие данные, корректировать хранимые данные без разрушения логических связей, установленных в схеме БД. Для этого схема БД должна быть правильно разработана, а операции ведения БД не должны нарушать схему БД.

Должна быть обеспечена означает защита данных от несанкционированного доступа, преднамеренного и непреднамеренного разрушения данных, хищения данных. 

Данными, хранящимися в БД должны пользоваться только лица, имеющие на это право и подтвердившие свои полномочия. Наиболее распространенным способом решения этой задачи является система паролей.

Каждый пользователь должен работать только с теми данными, которые необходимы для решения его задач, остальные данные должны быть для него «невидимыми». Каждому пользователю предоставляются определенные полномочия (привилегии) для работы с данными. Ему может быть предоставлено право только чтения из БД, право ввода в БД или право обновления и т.п. Все привилегии предоставляются только администратору БД. Обеспечение секретность данных. Секретные данные необходимо защищать от доступа системой специальных, достаточно сложных паролей. Сильно уязвимые данные следует шифровать.

Организация БД и методы доступа к данным должны обеспечивать высокую скорость обработки данных так, чтобы пользователь мог работать с БД в диалоговом режиме. Стоимость обслуживания пользователей не должна быть высокой.

Возможность выполнения этих требований определяется рядом факторов:

- Объемом хранимых данных;

- Быстродействием техники;

- Способом организации данных в БД;

- Решений, принимаемых разработчиками на этапе создания БД.

Представление данных в БД, сопровождающая документация, способ взаимодействия пользователя с БД должны удовлетворять определенным стандартам. Стандарты могут быть корпоративными, ведомственными, промышленными, национальными и международными. Соблюдение стандартов совершенно необходимо для совместного использования данных и для организации обмена данными между отдельными системами. Например, без принятия определенных стандартов нельзя было бы организовать сеть Internet.

\subsection{Назначение и основные компоненты системы баз данных}
Система БД включает два основных компонента: базу данных и систему управления базами данных – СУБД. Большинство СОД (среда общих данных) включают также программы обработки данных (прикладное программное обеспечение, ППО), которые обращаются к данным через СУБД.

СУБД обеспечивает выполнение двух групп функций:

- Предоставление доступа к базе данных прикладному программному обеспечению (или квалифицированным пользователям);

- Управление хранением и обработкой данных в БД.

Таким образом, обращение к базе данных возможно только через СУБД.

БД предназначена для хранения данных информационной системы. Пользователи обращаются к базе данных обычно не напрямую через средства СУБД, а с помощью внешнего интерфейса – приложения, входящего в состав АИС. Если пользователей можно разделить на группы по характеру решаемых задач, то приложений может быть несколько (по количеству задач или групп пользователей). 

\subsection{Уровни представления данных}
Современная технология баз данных основана на концепции многоуровневой архитектуры СУБД. Эти идеи впервые были сформулированы в отчёте рабочей группы по базам данных Комитета по планированию стандартов Американского национального института стандартов (ANSI/X3/SPARC). Этот отчёт был опубликован в 1975 г. В нём была предложена обобщенная трёхуровневая модель архитектуры СУБД, включающая концептуальный, внешний и внутренний уровни.

Концептуальный уровень архитектуры ANSI/SPARC служит для поддержки единого взгляда на базу данных, общего для всех её приложений и независимого от них и от среды хранения. Концептуальный уровень представляет собой формализованную информационно-логическую модель ПрО. Описание этого представления называется концептуальной схемой или схемой БД. Схема базы данных – это описание базы данных в терминах конкретной модели данных.

Внутренний уровень архитектуры поддерживает представление данных в среде хранения и пути доступа к ним. На этом архитектурном уровне БД представлена в полностью «материализованном» виде, тогда как на других уровнях идёт работа на уровне отдельных экземпляров или множества экземпляров данных. Описание БД на внутреннем уровне называется внутренней схемой или схемой хранения.

Внешний уровень архитектуры БД предназначен для групп пользователей. Описание представления данных для группы пользователей называется внешней схемой. Наличие внешнего уровня позволяет поддерживать разное представление одних и тех же данных для различных групп пользователей или задач.

Каждый из этих уровней может считаться управляемым, если он обладает внешним интерфейсом, который обеспечивает возможности определения данных. В этом случае становятся возможными формирование и системная поддержка независимого взгляда на БД для какой-либо группы персонала или пользователей, взаимодействующих с БД через интерфейс данного уровня.

В архитектурной модели ANSI/SPARC предполагается наличие в СУБД механизмов, обеспечивающих междууровневое отображение данных внешний – «концептуальный» и «концептуальный – внутренний». Функциональные возможности этих механизмов определяют степень независимости данных на всех уровнях. На переходе «внешний – концептуальный» обеспечивается логическая независимость данных, на переходе «концептуальный – внутренний» – физическая независимость. Под логической независимостью подразумевается возможность вносить изменения в концептуальный уровень, не меняя представление БД для пользователей, или изменять представление данных для пользователей без изменения концептуальной схемы. Физическая независимость данных подразумевает возможность вносить изменения в схему хранения, не меняя концептуальную схему БД.

Основной характеристикой баз данных является совместное использование данных многими пользователями АИС. Должно существовать какое-то общее понимание информации, представленной данными. Общее понимание должно относиться к чему-либо внешнему по отношению к пользователям, и оно должно быть зафиксировано. Для этого необходима некоторая предварительно определённая грамматика, которую принято называть моделью данных.
\subsection{Физическая организация БД}
Знание физической структуры данных позволяет обеспечить качественное выполнение физического проектирования БД.

Физическое проектирование БД — это отдельный процесс, тесно связанный с логическим проектированием и управлением размещения наборов данных, включающий процесс организации хранения данных с определением формата хранимой записи и классификации записей.

Реляционные СУБД (системы, которые реализуют реляционную модель работы с данными, основанную на связях (отношениях) между элементами информации) обладают рядом особенностей, влияющих на организацию внешней памяти. К наиболее важным особенностям можно отнести следующие:

1)	Наличие двух уровней системы: уровня непосредственного управления данными во внешней памяти (а также обычно управления буферами оперативной памяти, управления транзакциями и журнализацией изменений БД) и языкового уровня (например, уровня, реализующего язык SQL). При такой организации подсистема нижнего уровня должна поддерживать во внешней памяти набор базовых структур, конкретная интерпретация которых входит в число функций подсистемы верхнего уровня;

2)	Поддержание отношений-каталогов. Информация, связанная с именованием объектов базы данных и их конкретными свойствами (например, структура ключа индекса), поддерживается подсистемой языкового уровня.

С точки зрения структур внешней памяти отношение-каталог ничем не отличается от обычного отношения базы данных;

1) Регулярность структур данных. Поскольку основным объектом реляционной модели данных является плоская таблица, главный набор объектов внешней памяти может иметь очень простую регулярную структуру. При этом необходимо обеспечить возможность эффективного выполнения операторов языкового уровня как над одним отношением (простые селекция и проекция), так и над несколькими отношениями (наиболее распространено и трудоемко соединение нескольких отношений). Для этого во внешней памяти должны поддерживаться дополнительные «управляющие» структуры — индексы;

4)	Избыточность хранения данных для выполнения требования надежного хранения баз данных, что обычно реализуется в виде журнала изменений базы данных.

Соответственно возникают следующие разновидности объектов баз данных:

- Таблицы — основные объекты базы данных, большей частью непосредственно видимые пользователям;

- Последовательности — объекты БД, используемые для формирования уникальных числовых величин;

- Индексы — управляющие структуры, создаваемые по инициативе разработчика (администратора) баз данных или верхнего уровня системы в целях повышения эффективности выполнения запросов и обычно автоматически поддерживаемые нижним уровнем системы;

- Представления (views) — хранимые предложения SQL (запросы на выборку), которые можно запросить как таблицу;

- Триггеры (triggers) — хранимые процедуры, запускаемые при выполнении определенных действий с таблицей;

- Хранимая процедура — выполняемый объект, реализованный с помощью процедурного расширения SQL, которому можно передать аргументы и получить от него сформированные результаты;

- Хранимая функция отличается от хранимой процедуры тем, что возвращаемым результатом выполнения функции является некоторое единичное значение;

- Хранимые пакеты представляют собой совокупность процедур, переменных и функций, объединенных для выполнения некоторой задачи;

- Журнальная информация, поддерживаемая для удовлетворения потребности в надежном хранении данных;

- Служебная информация, поддерживаемая для удовлетворения внутренних потребностей нижнего уровня системы (например, информация о связях между таблицами).
\subsection{Организация индексов, методы хранения и доступа к данным}
Во всех существующих на рынке СУБД имеется в наличии средство, оптимизирующее дисковое пространство для хранения данных, а также обеспечивающее оптимальный по скорости доступ к данным. Такая надстройка над данными называется индексами (некий упорядоченный указатель на записи в таблице). Понятие «указатель» означает, что индекс представляется как совокупность значений одного или нескольких полей таблицы БД и адресов страниц данных, где физически располагаются эти значения. То есть индекс состоит из пар значений «значение поля» — «физическое расположение этого поля». При этом индекс не является частью таблицы — это отдельный, взаимосвязанный с таблицей (или таблицами) объект БД. В целом индекс можно описать как специальную структуру данных, создаваемую автоматически или по запросу пользователя.

 Поиск данных в таблице без использования индекса можно сравнить с последовательным перебором всех книг в библиотеке. Большинство таблиц в БД имеют большое количество записей, которые хранятся в определенном формате, и поиск необходимых данных по заданному критерию запроса путем последовательного перебора таблицы — запись за записью, естественно, может занимать большое количество времени. Индекс позволяет быстро искать строки, удовлетворяющие критерию поиска. Ускорение работы с использованием индексов обеспечивается несколькими факторами, во-первых, за счёт того, что индекс имеет специальную структуру, оптимизированную под поиск, во-вторых, сами таблицы в БД могут храниться таким образом, чтобы обеспечивать оптимальный доступ к индексируемым полям.

Фактически, индекс описывает отношения упорядочивания и однозначности значений, с помощью которых обеспечивается эффективный доступ к записям в таблицах базы данных. При этом следует отметить, что как бы ни были организованы индексы, их назначение состоит в обеспечении эффективного доступа к записи таблицы по некоторому ключу.

Общей идеей любой организации индекса, поддерживающего прямой доступ по ключу и последовательный просмотр в порядке возрастания или убывания значений ключа, является хранение упорядоченного списка значений ключа с привязкой к каждому значению ключа списка идентификаторов кортежей. Один вид организации индекса отличается от другого главным образом по способу поиска ключа с заданным значением. 

Существуют различные методы хранения и доступа к данным:

1) Инвертированный метод (вторичное индексирование);

2) Прямой метод доступа и хеширование;

3) Двоичный масочный индекс (Bitmap);

4) Кластерный индекс.

В инвертированном методе для каждого значения в указанном поле создается отдельный список идентификаторов записей, которые содержат это значение. Эти списки образуют индекс. Используется только для выборки данных, а не для физического упорядочивания записей. Позволяет эффективно выполнять сложные запросы с использованием операций объединения и пересечения условий. Эффективность зависит от объема данных, количества уровней индекса и распределения памяти.

Прямой метод доступа основан на прямом соответствии между ключом записи и ее физическим адресом, что позволяет получить доступ к данным за одно обращение. Метод хеширования является разновидностью прямого доступа. Физический адрес вычисляется с помощью хеш-функции, которая преобразует ключ в адрес.

В двоичном масочном индексе (Bitmap) для каждого допустимого значения столбца создается битовая маска. Бит устанавливается в 1, если запись содержит соответствующее значение. Особенно эффективен для столбцов с небольшим количеством возможных значений. Обеспечивает высокую скорость выполнения запросов с условиями AND, OR, NOT.

В кластерном индексе записи в таблице физически упорядочиваются на диске в последовательности, соответствующей значениям ключа индекса. Таким образом, сам индекс определяет порядок хранения данных. Обеспечивает очень быстрый доступ к данным по диапазону значений, так как связанные записи хранятся рядом. В таблице может быть только один кластерный индекс, поскольку данные могут быть физически упорядочены только одним способом.

\subsection{Оптимизация работы с базами данных}
Можно дать некоторые рекомендации, которые позволят добиться повышения быстродействия и уберегут разработчиков баз данных от ошибок, которые могут возникнуть при организации и разработке баз данных:

1) Создавайте таблицы, не содержащие избыточных данных, — стремитесь к нормализации;

2) Создавайте индексы для сортируемых и объединяемых полей, а также для полей, используемых при задании критериев запроса в SQL-запросах. Повышение быстродействия при выполнении SQL-запросов можно достичь индексацией полей, являющихся внешними ключами;

3) Определяйте тип данных полей с учетом максимально точно подходящего типа данных. Это поможет уменьшить размеры базы данных и увеличит скорость выполнения операций связи. При описании поля следует задать для него тип данных наименьшего размера, позволяющий хранить нужные данные:

При выборе типа данных, на котором определяется поле, следует учитывать:

- Тип значений, которые должны отображаться в поле (например, нельзя хранить текст в поле, имеющем числовой тип данных);

- Размер данных для хранения значений в поле;

- Возможность применения математических и других операций со значениями в поле (например, суммировать значения можно в числовых полях и в полях, имеющих валютный формат, а значения в текстовых полях и полях объектов OLE — нельзя);

- Необходимость сортировки или индексирования поля (сортировать и индексировать поля MЕМО, гиперссылки и объекты OLE невозможно);

- Необходимость использования полей в группировке записей в запросах или отчетах. Поля MЕМО, гиперссылки и объекты OLE использовать для группировки записей нельзя;

- Порядок сортировки значений в поле.

Числа в текстовых полях сортируются как строки чисел (1, 10, 100, 2, 20, 200 и т. д.), а не как числовые значения. Для сортировки чисел как числовых значений необходимо использовать числовые поля или поля, имеющие денежный формат (если СУБД поддерживает такой тип данных). Также многие форматы дат невозможно отсортировать надлежащим образом, если они были введены в текстовое поле.

Поля с типом данных объект OLE используются для хранения таких данных, как документы Microsoft Word или Microsoft Excel, рисунки, звук и объекты других программ. Объекты OLE могут быть связаны или внедрены в поля таблиц СУБД, поддерживающих возможность работы с OLE-объектами.

Первичные и внешние ключи следует по возможности определять только на числовых полях либо на полях типа Дата/время, если это поле входит в составной первичный ключ.

Если первичный ключ может быть построен не менее чем на четырех полях — следует заменить его суррогатным ключом.

Таблицы справочники-классификаторы создавайте только для реально повторяющихся значений — например, нет смысла в базе данных создавать отдельные справочники-классификаторы для фамилий, имен и отчеств, достаточно в таблице с описанием людей выделить три отдельных поля: Фамилия, Имя, Отчество.

Необходимо периодически производить сжатие базы данных. При наличии запоминающих устройств с большим объемом памяти проблема сжатия данных все же не утратила своей актуальности. Действительно, с приходом новых технологий появилась возможность создания БД с большим объемом хранимой в них информации (например, распределенные БД с таблицами, содержащими гигабайты данных), но для хранения таких БД по-прежнему приходится применять технологию сжатия данных.

Естественно, что механизм сжатия данных должен быть обратим. Преимущества СУБД, использующих сжатие данных:

1) В территориально удаленных СУБД передача данных от одного узла к другому требует меньше времени;

2) Обеспечивает более высокую скорость передачи данных по сравнению с несжатыми данными.

При неавтоматической репликации данных (работы с копией БД или объектами, допускающими синхронизацию данных) возможно использование обычных файловых архиваторов:

1) Для хранения сжатых данных при резервном копировании требуется меньше устройств резервного копирования;

2) При использовании сжатия данных появляется возможность упаковывать больше ключей в блок индекса заданного размера.

Используемые значения ключей сначала сжимаются, а уже потом начинают сравниваться со сжатыми ключами в самом индексе. Следовательно, если мы имеем больше ключей, хранимых в индексном блоке заданного размера, то в результате потребуется меньше операций для поиска того блока индекса, который необходим для доступа к нужным данным.

В различных СУБД могут существовать свои алгоритмы сжатия данных, однако не существует обобщающего алгоритма для обеспечения наилучшего эффекта сжатия данных. Так, например, в СУБД MS Access при сжатии базы данных индексы оптимизируются по быстродействию, т. е. для поддержания оптимизации по быстродействию необходимо регулярно выполнять сжатие базы данных. Для такой цели в этой СУБД существует специальная подпрограмма сжатия данных.

Следует удалять индексы, необходимость в которых отсутствует. Используйте буферы оперативной памяти для временного хранения данных.

В настоящее время существуют СУБД, способные обрабатывать данные в оперативной памяти на качественно высоком уровне. Использование СУБД такого класса позволяет пользователям обрабатывать данные в несколько раз быстрее, чем в случае с работой при обращении непосредственно к жестким дискам. Обычно для БД, поддерживаемых в оперативной памяти, их состояние сохраняется в некоторых контрольных точках в виде дисковых копий. Такие контрольные точки возникают в периоды наименьшей активности пользователей.
\newpage
\section{Отличие реляционных и нереляционных моделей данных}
\section{Основные функции и возможности систем управления базами данных}
\section{Язык SQL как инструмент взаимодействия с базами данных}
\section{Концепции целостности и надежности в системах хранения данных}
\section{Облачные подходы и онлайн-инструменты для работы с базами данных}

\end{document}