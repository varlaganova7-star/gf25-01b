\documentclass{ou}
\begin{document}
\thispagestyle{empty}
\begin{center}
\minobrRF\\
\FGAVO\\
\textbf{\sixteen{\university}}\\
\institute\\
\department\\[1\baselineskip]
{\fontsize{16}{24}\selectfont\bfseries \work}\\[0.5\baselineskip]
\discipline\\
Технологии создания и обработки табличных данных\\[1\baselineskip]
\end{center}
\instructor \hfill \selectfont \Nigmatullin\\[1\baselineskip]
\students \hfill \selectfont \group\\
\vfill
\begin{center}
    \city
\end{center}
\newpage
\tableofcontents
\newpage
\section{Общие принципы организации и хранения данных в базах данных}
База данных (БД) – совокупность данных, организованных по определённым правилам, предусматривающим общие принципы описания, хранения и манипулирования данными, независимая от прикладных программ. Эти данные относятся к определённой предметной области и организованы таким образом, что могут быть использованы для решения многих задач многими пользователями. 

Система управления базами данных (СУБД) – это специальный вид программного обеспечения, который позволяет управлять и организовывать данные. В отличие от простого хранения данных в файловой системе, СУБД предоставляет удобные механизмы для поиска, обновления, добавления и удаления данных. 

Базы данных нужны для хранения больших объёмов структурированных данных, где важны безопасность, надёжность и возможность интеграции с другими системами. Хранение данных в файлах имеет недостатки, которые делают их менее эффективными для сложных задач.

\subsection{Организация данных}
К организации данных в системах автоматизированной обработки информации возможны два подхода: 

1) Каждый пользователь системы создает наборы данных, необходимых для решения его задач, и пишет программы обработки данных. Например, в рамках ВУЗа различные подразделения (деканат, отдел кадров, бухгалтерия и т.п.) могут создать свои подсистемы, предназначенные для решения определенных задач;

2) Вся информация, описывающая определенную предметную область, хранится, интегрировано, в единой базе данных (БД) и каждый из пользователей имеет доступ к тем данным, которые необходимы ему для решения его задач;

Первый из подходов имеет ряд недостатков:

- В различных подсистемах часто хранится одна и та же информация (сведения о студентах, о преподавателях и т.п.), т.е. данные дублируются, и возникает избыточность информации. При появлении изменений в данных необходимо обновлять многочисленные наборы данных и, если отдельные наборы окажутся не скорректированы, возникнет противоречивость данных;

- Обмен данными между отдельными подсистемами затруднен или невозможен, т.к. прикладные программы отдельных подсистем написаны на различных языках программирования, а данные представлены в различных форматах;

- При появлении в подсистеме новых задач, а, следовательно, и новых данных придется вносить изменения в уже созданные файлы и программы, т.к. данные описаны в каждой из прикладных программ (описаны типы и форматы данных, типы файлов). В подобном случае говорят, что прикладные программы зависят от хранимых данных;

Существенным достоинством первого подхода является наличие у каждого набора данных единственного владельца, что снижает риск неавторизованного доступа к данным, их искажения и разрушения.

При хранении данных в БД перечисленные недостатки снимаются. Однако в этом случае возникает другой недостаток: у данных нет единого хозяина. Из-за этого снижается ответственность за правильность хранимых данных и нарушается секретность. Для устранения этого недостатка для БД разрабатывается специальная система защиты.

Один из основных факторов построения базы данных заключается в том, что она должна соответствовать специфическим требованиям конкретной задачи или приложения. Для этого необходимо провести анализ данных, определить структуру и свойства объектов, которые будут храниться в базе данных.

Принцип построения базы данных состоит в том, чтобы организовать данные в такой форме, чтобы они были легко доступны и могли быть обработаны с помощью компьютера. База данных представляет собой структурированную коллекцию данных, которая хранится в компьютере и доступна для использования.

Для построения базы данных необходимо определить ее цели и требования к данным. Например, если база данных предназначена для хранения информации о клиентах, то необходимо определить, какие данные о клиентах будут храниться, как они будут структурированы и каким образом они будут связаны между собой.

При проектировании базы данных необходимо учитывать следующие принципы:

1) Нормализация данных. Данные должны быть разбиты на отдельные таблицы и связаны между собой отношениями. Это позволяет уменьшить дублирование данных и обеспечить их целостность;

2) Соответствие типов данных. Каждый столбец в таблице должен иметь соответствующий тип данных, чтобы избежать ошибок при работе с данными;

3) Определение первичного ключа. Каждая таблица должна иметь уникальный идентификатор, который позволяет однозначно идентифицировать каждую запись в таблице;

4) Установление связей между таблицами. Данные в разных таблицах должны быть связаны между собой отношениями, чтобы обеспечить целостность данных и возможность их последующей обработки;

5) Определение индексов. Индексы позволяют ускорить поиск данных в таблицах и улучшить производительность базы данных;

6) Определение ограничений на данные. Ограничения позволяют контролировать ввод данных и обеспечить их правильность и целостность.

В основе построения БД лежат определенные научные принципы, позволяющие создавать высококачественные системы, отвечающие современным требованиям. Из множества используемых принципов создания БД выделяются наиболее существенные:

- Интеграции данных;

- Централизации управления данными.

Оба принципа отражают суть БД. Интеграция является основой организации БД, централизация управления – основой организации и функционирования СУБД. 

Суть принципа интеграции данных состоит в объединении отдельных, взаимно не связанных данных в единое целое, в роли которого выступает база данных, в результате чего пользователю и его прикладным программам все данные представляются единым информационным массивом. Следование принципу интеграции обеспечивает:

- Упрощение поиска взаимосвязанных данных и их совместную обработку;

- Уменьшение избыточности данных;

- Упрощение процесса ведения БД.

Принцип централизации управления состоит в передаче всех функций управления данными единому комплексу управляющих программ – СУБД.

\subsection{Требования, предъявляемые к БД}
Данные в БД не должны дублироваться. Избыточность данных, если она существует, влечет две опасности: 

1) Неоправданно большой расход памяти и уменьшение времени отклика системы при обработке излишне больших объемов данных;

2) Нарушение непротиворечивости данных, т.е. возникновение такой ситуации, когда в различных местах машинной памяти хранятся противоречивые данные.

Противоречивость может возникнуть в результате корректировки избыточных данных. При внесении изменений в логическую запись может случиться так, что отдельные экземпляры этой записи, хранящиеся в различных местах машинной памяти, окажутся нескорректированы. Противоречивость может возникнуть и при корректировке не избыточных данных.

В БД должны храниться только правильные данные, т.е. соблюдаются логические условия, в соответствии с которыми данные считаются правильными. Разрушение и искажение данных возможно в результате неосторожных действий пользователей, в результате ошибок в программах и сбоев оборудования. 

Для обеспечения целостности на данные, хранящиеся в БД, накладывают ограничения. При этом определяются условия, которым должны соответствовать значения данных. Например, один и тот же служащий не может иметь два различных года рождения и т.п.. Подобные ограничения называются законами БД. Выполнимость законов БД периодически проверяется СУБД. 

Для предотвращения возможности ввода неправильных данных разрабатываются средства контроля правильности вводимых данных. Например, можно использовать процедуры, проверяющие принадлежность вводимых значений определенному диапазону допустимых значений. Например, количество рабочих дней ограничивается сверху количеством дней в текущем месяце.

Целостность данных может нарушиться при неудачном завершении транзакции. Транзакцией называется некоторая неделимая последовательность операций над данными, выполняемая по одному запросу к БД. Примером транзакции является операция перевода денег с одного счета на другой в банковской системе. Здесь необходимо последовательное выполнение нескольких операций. Деньги снимаются с одного счета, данные корректируются, затем деньги добавляются к другому счету и данные вновь корректируются. Если хотя бы одно из действий не выполняется успешно, результат транзакции окажется неверным. СУБД должна отслеживать ход выполнения транзакции от начала до ее завершения. Если по какой-то причине какая-либо из операций не выполнилась, то транзакция отменяется полностью. При этом выполняется «откат» путем отмены всех уже выполненных изменений.

В БД должны быть предусмотрены средства восстановления данных после программных сбоев и сбоев оборудования. Существуют программы создания резервных копий и специальные программы, которые автоматически фиксируют любые внесенные в БД изменения (создается файл корректур). Если текущая версии БД испорчена, то берется предыдущая версия, в нее вносятся изменения, зафиксированные в файле корректур, и текущее (актуальное) состояние БД восстанавливается. 

Различные СУБД в той или иной мере располагают средствами обеспечения целостности данных. В противном случае такие средства разрабатываются системным программистом.

Прикладные программы не должны зависеть от хранимых данных, т.е. от способа хранения данных в физической памяти. Это позволяет добавлять в БД новые данные, изменять структуры хранения данных, создавать на БД новые приложения. Ранее созданные программы при этом не должны «чувствовать» эти изменения.  

Структура БД должна позволять включать новые и удалять устаревшие данные, корректировать хранимые данные без разрушения логических связей, установленных в схеме БД. Для этого схема БД должна быть правильно разработана, а операции ведения БД не должны нарушать схему БД.

Должна быть обеспечена означает защита данных от несанкционированного доступа, преднамеренного и непреднамеренного разрушения данных, хищения данных. 

Данными, хранящимися в БД должны пользоваться только лица, имеющие на это право и подтвердившие свои полномочия. Наиболее распространенным способом решения этой задачи является система паролей.

Каждый пользователь должен работать только с теми данными, которые необходимы для решения его задач, остальные данные должны быть для него «невидимыми». Каждому пользователю предоставляются определенные полномочия (привилегии) для работы с данными. Ему может быть предоставлено право только чтения из БД, право ввода в БД или право обновления и т.п. Все привилегии предоставляются только администратору БД. Обеспечение секретность данных. Секретные данные необходимо защищать от доступа системой специальных, достаточно сложных паролей. Сильно уязвимые данные следует шифровать.

Организация БД и методы доступа к данным должны обеспечивать высокую скорость обработки данных так, чтобы пользователь мог работать с БД в диалоговом режиме. Стоимость обслуживания пользователей не должна быть высокой.

Возможность выполнения этих требований определяется рядом факторов:

- Объемом хранимых данных;

- Быстродействием техники;

- Способом организации данных в БД;

- Решений, принимаемых разработчиками на этапе создания БД.

Представление данных в БД, сопровождающая документация, способ взаимодействия пользователя с БД должны удовлетворять определенным стандартам. Стандарты могут быть корпоративными, ведомственными, промышленными, национальными и международными. Соблюдение стандартов совершенно необходимо для совместного использования данных и для организации обмена данными между отдельными системами. Например, без принятия определенных стандартов нельзя было бы организовать сеть Internet.

\subsection{Назначение и основные компоненты системы баз данных}
Система БД включает два основных компонента: базу данных и систему управления базами данных – СУБД. Большинство СОД (среда общих данных) включают также программы обработки данных (прикладное программное обеспечение, ППО), которые обращаются к данным через СУБД.

СУБД обеспечивает выполнение двух групп функций:

- Предоставление доступа к базе данных прикладному программному обеспечению (или квалифицированным пользователям);

- Управление хранением и обработкой данных в БД.

Таким образом, обращение к базе данных возможно только через СУБД.

БД предназначена для хранения данных информационной системы. Пользователи обращаются к базе данных обычно не напрямую через средства СУБД, а с помощью внешнего интерфейса – приложения, входящего в состав АИС. Если пользователей можно разделить на группы по характеру решаемых задач, то приложений может быть несколько (по количеству задач или групп пользователей). 

\subsection{Уровни представления данных}
Современная технология баз данных основана на концепции многоуровневой архитектуры СУБД. Эти идеи впервые были сформулированы в отчёте рабочей группы по базам данных Комитета по планированию стандартов Американского национального института стандартов (ANSI/X3/SPARC). Этот отчёт был опубликован в 1975 г. В нём была предложена обобщенная трёхуровневая модель архитектуры СУБД, включающая концептуальный, внешний и внутренний уровни.

Концептуальный уровень архитектуры ANSI/SPARC служит для поддержки единого взгляда на базу данных, общего для всех её приложений и независимого от них и от среды хранения. Концептуальный уровень представляет собой формализованную информационно-логическую модель ПрО. Описание этого представления называется концептуальной схемой или схемой БД. Схема базы данных – это описание базы данных в терминах конкретной модели данных.

Внутренний уровень архитектуры поддерживает представление данных в среде хранения и пути доступа к ним. На этом архитектурном уровне БД представлена в полностью «материализованном» виде, тогда как на других уровнях идёт работа на уровне отдельных экземпляров или множества экземпляров данных. Описание БД на внутреннем уровне называется внутренней схемой или схемой хранения.

Внешний уровень архитектуры БД предназначен для групп пользователей. Описание представления данных для группы пользователей называется внешней схемой. Наличие внешнего уровня позволяет поддерживать разное представление одних и тех же данных для различных групп пользователей или задач.

Каждый из этих уровней может считаться управляемым, если он обладает внешним интерфейсом, который обеспечивает возможности определения данных. В этом случае становятся возможными формирование и системная поддержка независимого взгляда на БД для какой-либо группы персонала или пользователей, взаимодействующих с БД через интерфейс данного уровня.

В архитектурной модели ANSI/SPARC предполагается наличие в СУБД механизмов, обеспечивающих междууровневое отображение данных внешний – «концептуальный» и «концептуальный – внутренний». Функциональные возможности этих механизмов определяют степень независимости данных на всех уровнях. На переходе «внешний – концептуальный» обеспечивается логическая независимость данных, на переходе «концептуальный – внутренний» – физическая независимость. Под логической независимостью подразумевается возможность вносить изменения в концептуальный уровень, не меняя представление БД для пользователей, или изменять представление данных для пользователей без изменения концептуальной схемы. Физическая независимость данных подразумевает возможность вносить изменения в схему хранения, не меняя концептуальную схему БД.

Основной характеристикой баз данных является совместное использование данных многими пользователями АИС. Должно существовать какое-то общее понимание информации, представленной данными. Общее понимание должно относиться к чему-либо внешнему по отношению к пользователям, и оно должно быть зафиксировано. Для этого необходима некоторая предварительно определённая грамматика, которую принято называть моделью данных.
\subsection{Физическая организация БД}
Знание физической структуры данных позволяет обеспечить качественное выполнение физического проектирования БД.

Физическое проектирование БД — это отдельный процесс, тесно связанный с логическим проектированием и управлением размещения наборов данных, включающий процесс организации хранения данных с определением формата хранимой записи и классификации записей.

Реляционные СУБД (системы, которые реализуют реляционную модель работы с данными, основанную на связях (отношениях) между элементами информации) обладают рядом особенностей, влияющих на организацию внешней памяти. К наиболее важным особенностям можно отнести следующие:

1)	Наличие двух уровней системы: уровня непосредственного управления данными во внешней памяти (а также обычно управления буферами оперативной памяти, управления транзакциями и журнализацией изменений БД) и языкового уровня (например, уровня, реализующего язык SQL). При такой организации подсистема нижнего уровня должна поддерживать во внешней памяти набор базовых структур, конкретная интерпретация которых входит в число функций подсистемы верхнего уровня;

2)	Поддержание отношений-каталогов. Информация, связанная с именованием объектов базы данных и их конкретными свойствами (например, структура ключа индекса), поддерживается подсистемой языкового уровня.

С точки зрения структур внешней памяти отношение-каталог ничем не отличается от обычного отношения базы данных. Основным объектом реляционной модели данных является плоская таблица, главный набор объектов внешней памяти может иметь очень простую регулярную структуру. При этом необходимо обеспечить возможность эффективного выполнения операторов языкового уровня как над одним отношением (простые селекция и проекция), так и над несколькими отношениями (наиболее распространено и трудоемко соединение нескольких отношений). Для этого во внешней памяти должны поддерживаться дополнительные «управляющие» структуры — индексы.

Соответственно возникают следующие разновидности объектов баз данных:

1) Таблицы — основные объекты базы данных, большей частью непосредственно видимые пользователям;

2) Последовательности — объекты БД, используемые для формирования уникальных числовых величин;

3) Индексы — управляющие структуры, создаваемые по инициативе разработчика (администратора) баз данных или верхнего уровня системы в целях повышения эффективности выполнения запросов и обычно автоматически поддерживаемые нижним уровнем системы;

4) Представления (views) — хранимые предложения SQL (запросы на выборку), которые можно запросить как таблицу;

5) Триггеры (triggers) — хранимые процедуры, запускаемые при выполнении определенных действий с таблицей;

6) Хранимая процедура — выполняемый объект, реализованный с помощью процедурного расширения SQL, которому можно передать аргументы и получить от него сформированные результаты;

7) Хранимая функция отличается от хранимой процедуры тем, что возвращаемым результатом выполнения функции является некоторое единичное значение;

8) Хранимые пакеты представляют собой совокупность процедур, переменных и функций, объединенных для выполнения некоторой задачи;

9) Журнальная информация, поддерживаемая для удовлетворения потребности в надежном хранении данных;

10) Служебная информация, поддерживаемая для удовлетворения внутренних потребностей нижнего уровня системы (например, информация о связях между таблицами).
\subsection{Организация индексов, методы хранения и доступа к данным}
Во всех существующих на рынке СУБД имеется в наличии средство, оптимизирующее дисковое пространство для хранения данных, а также обеспечивающее оптимальный по скорости доступ к данным. Такая надстройка над данными называется индексами (некий упорядоченный указатель на записи в таблице). Понятие «указатель» означает, что индекс представляется как совокупность значений одного или нескольких полей таблицы БД и адресов страниц данных, где физически располагаются эти значения. То есть индекс состоит из пар значений «значение поля» — «физическое расположение этого поля». При этом индекс не является частью таблицы — это отдельный, взаимосвязанный с таблицей (или таблицами) объект БД. В целом индекс можно описать как специальную структуру данных, создаваемую автоматически или по запросу пользователя.

Поиск данных в таблице без использования индекса можно сравнить с последовательным перебором всех книг в библиотеке. Большинство таблиц в БД имеют большое количество записей, которые хранятся в определенном формате, и поиск необходимых данных по заданному критерию запроса путем последовательного перебора таблицы — запись за записью, естественно, может занимать большое количество времени. Индекс позволяет быстро искать строки, удовлетворяющие критерию поиска. Ускорение работы с использованием индексов обеспечивается несколькими факторами, во-первых, за счёт того, что индекс имеет специальную структуру, оптимизированную под поиск, во-вторых, сами таблицы в БД могут храниться таким образом, чтобы обеспечивать оптимальный доступ к индексируемым полям.

Фактически, индекс описывает отношения упорядочивания и однозначности значений, с помощью которых обеспечивается эффективный доступ к записям в таблицах базы данных. При этом следует отметить, что как бы ни были организованы индексы, их назначение состоит в обеспечении эффективного доступа к записи таблицы по некоторому ключу.

Общей идеей любой организации индекса, поддерживающего прямой доступ по ключу и последовательный просмотр в порядке возрастания или убывания значений ключа, является хранение упорядоченного списка значений ключа с привязкой к каждому значению ключа списка идентификаторов кортежей. Один вид организации индекса отличается от другого главным образом по способу поиска ключа с заданным значением. 

Существуют различные методы хранения и доступа к данным:

- Инвертированный метод (вторичное индексирование);

- Прямой метод доступа и хеширование;

- Двоичный масочный индекс (Bitmap);

- Кластерный индекс.

В инвертированном методе для каждого значения в указанном поле создается отдельный список идентификаторов записей, которые содержат это значение. Эти списки образуют индекс. Используется только для выборки данных, а не для физического упорядочивания записей. Позволяет эффективно выполнять сложные запросы с использованием операций объединения и пересечения условий. Эффективность зависит от объема данных, количества уровней индекса и распределения памяти.

Прямой метод доступа основан на прямом соответствии между ключом записи и ее физическим адресом, что позволяет получить доступ к данным за одно обращение. Метод хеширования является разновидностью прямого доступа. Физический адрес вычисляется с помощью хеш-функции, которая преобразует ключ в адрес.

В двоичном масочном индексе (Bitmap) для каждого допустимого значения столбца создается битовая маска. Бит устанавливается в 1, если запись содержит соответствующее значение. Особенно эффективен для столбцов с небольшим количеством возможных значений. Обеспечивает высокую скорость выполнения запросов с условиями AND, OR, NOT.

В кластерном индексе записи в таблице физически упорядочиваются на диске в последовательности, соответствующей значениям ключа индекса. Таким образом, сам индекс определяет порядок хранения данных. Обеспечивает очень быстрый доступ к данным по диапазону значений, так как связанные записи хранятся рядом. В таблице может быть только один кластерный индекс, поскольку данные могут быть физически упорядочены только одним способом.

\subsection{Оптимизация работы с базами данных}
Можно дать некоторые рекомендации, которые позволят добиться повышения быстродействия и уберегут разработчиков баз данных от ошибок, которые могут возникнуть при организации и разработке баз данных:

1) Создавайте таблицы, не содержащие избыточных данных, — стремитесь к нормализации;

2) Создавайте индексы для сортируемых и объединяемых полей, а также для полей, используемых при задании критериев запроса в SQL-запросах. Повышение быстродействия при выполнении SQL-запросов можно достичь индексацией полей, являющихся внешними ключами;

3) Определяйте тип данных полей с учетом максимально точно подходящего типа данных. Это поможет уменьшить размеры базы данных и увеличит скорость выполнения операций связи. При описании поля следует задать для него тип данных наименьшего размера, позволяющий хранить нужные данные:

При выборе типа данных, на котором определяется поле, следует учитывать:

1) Тип значений, которые должны отображаться в поле (например, нельзя хранить текст в поле, имеющем числовой тип данных);

2) Размер данных для хранения значений в поле;

3) Возможность применения математических и других операций со значениями в поле (например, суммировать значения можно в числовых полях и в полях, имеющих валютный формат, а значения в текстовых полях и полях объектов OLE — нельзя);

4) Необходимость сортировки или индексирования поля (сортировать и индексировать поля MЕМО, гиперссылки и объекты OLE невозможно);

5) Необходимость использования полей в группировке записей в запросах или отчетах. Поля MЕМО, гиперссылки и объекты OLE использовать для группировки записей нельзя;

6) Порядок сортировки значений в поле.

Числа в текстовых полях сортируются как строки чисел (1, 10, 100, 2, 20, 200 и т. д.), а не как числовые значения. Для сортировки чисел как числовых значений необходимо использовать числовые поля или поля, имеющие денежный формат (если СУБД поддерживает такой тип данных). Также многие форматы дат невозможно отсортировать надлежащим образом, если они были введены в текстовое поле.

Поля с типом данных объект OLE используются для хранения таких данных, как документы Microsoft Word или Microsoft Excel, рисунки, звук и объекты других программ. Объекты OLE могут быть связаны или внедрены в поля таблиц СУБД, поддерживающих возможность работы с OLE-объектами.

Первичные и внешние ключи следует по возможности определять только на числовых полях либо на полях типа Дата/время, если это поле входит в составной первичный ключ.

Если первичный ключ может быть построен не менее чем на четырех полях — следует заменить его суррогатным ключом.

Таблицы справочники-классификаторы создавайте только для реально повторяющихся значений — например, нет смысла в базе данных создавать отдельные справочники-классификаторы для фамилий, имен и отчеств, достаточно в таблице с описанием людей выделить три отдельных поля: Фамилия, Имя, Отчество.

Необходимо периодически производить сжатие базы данных. При наличии запоминающих устройств с большим объемом памяти проблема сжатия данных все же не утратила своей актуальности. Действительно, с приходом новых технологий появилась возможность создания БД с большим объемом хранимой в них информации (например, распределенные БД с таблицами, содержащими гигабайты данных), но для хранения таких БД по-прежнему приходится применять технологию сжатия данных.

Естественно, что механизм сжатия данных должен быть обратим. Преимущества СУБД, использующих сжатие данных:

1) В территориально удаленных СУБД передача данных от одного узла к другому требует меньше времени;

2) Обеспечивает более высокую скорость передачи данных по сравнению с несжатыми данными.

При неавтоматической репликации данных (работы с копией БД или объектами, допускающими синхронизацию данных) возможно использование обычных файловых архиваторов:

1) Для хранения сжатых данных при резервном копировании требуется меньше устройств резервного копирования;

2) При использовании сжатия данных появляется возможность упаковывать больше ключей в блок индекса заданного размера.

Используемые значения ключей сначала сжимаются, а уже потом начинают сравниваться со сжатыми ключами в самом индексе. Следовательно, если мы имеем больше ключей, хранимых в индексном блоке заданного размера, то в результате потребуется меньше операций для поиска того блока индекса, который необходим для доступа к нужным данным.

В различных СУБД могут существовать свои алгоритмы сжатия данных, однако не существует обобщающего алгоритма для обеспечения наилучшего эффекта сжатия данных. Так, например, в СУБД MS Access при сжатии базы данных индексы оптимизируются по быстродействию, т. е. для поддержания оптимизации по быстродействию необходимо регулярно выполнять сжатие базы данных. Для такой цели в этой СУБД существует специальная подпрограмма сжатия данных.

Следует удалять индексы, необходимость в которых отсутствует. Используйте буферы оперативной памяти для временного хранения данных.

В настоящее время существуют СУБД, способные обрабатывать данные в оперативной памяти на качественно высоком уровне. Использование СУБД такого класса позволяет пользователям обрабатывать данные в несколько раз быстрее, чем в случае с работой при обращении непосредственно к жестким дискам. Обычно для БД, поддерживаемых в оперативной памяти, их состояние сохраняется в некоторых контрольных точках в виде дисковых копий. Такие контрольные точки возникают в периоды наименьшей активности пользователей.
\newpage
\section{Отличие реляционных и нереляционных моделей данных}
Современные информационные технологии требуют эффективных методов хранения, обработки и анализа данных.
\subsection{Типы и структуры данных}
Современные компании работают с тремя основными типами данных, каждый из которых требует определенных подходов к хранению и управлению:

1) Структурированные данные -- традиционная форма бизнес-информации, идеальна для бизнес-операций и отчетности, помещается в организованные таблицы с определенными строками и столбцами (реляционные базы данных);

2) Полуструктурированные данные -- информация, которая имеет некоторые организационные элементы, но не вписывается в жесткие таблицы базы данных (лучше всего обрабатывается моделью реляционной базы данных), этот тип сочетает в себе как организованные, так и произвольные элементы, сохраняя при этом достаточную структуру для эффективного анализа;

3) Неструктурированные данные -- информация без заранее определенной организации -- посты в социальных сетях, изображения, видео и отзывы клиентов, этот тип требует специализированных решений для хранения данных и гибкой модели данных (обычно нереляционной) для извлечения значимых аналитических сведений.

Системы управления базами данных (СУБД)
Системы баз данных (СУБД) – обеспечивают возможность структурированного хранения информации, а также ее быстрый доступ. Сама база данных - это просто организованные данные, хранящиеся на диске - вы не можете "увидеть" их напрямую. Представьте себе содержимое картотечного шкафа, запертого в запечатанной комнате. Именно здесь на помощь приходит система управления базами данных (СУБД).

СУБД выступает в качестве способа для:

- Просмотра данных (например, открытие картотечного шкафа);

- Работы с данными (например, добавление, изменение или удаление файлов);

- Защиты данных (например, с помощью замка и ключа);

- Сортировки данных (например, с помощью меток и папок).

В аналитике данных важную роль играют как реляционные, так и нереляционные базы данных, каждая из которых имеет свои особенности и применимость в зависимости от задач.

Реляционные и нереляционные базы данных представляют собой два различных способа хранения данных и управления ими.

\subsection{Реляционные базы данных (SQL)}
Реляционные базы данных (РБД) являются основой для хранения структурированных данных. Эти системы построены на основе таблиц, каждая из которых состоит из строк и столбцов, что делает данные легко доступными для обработки с помощью языка запросов SQL (Structured Query Language). Они отлично справляются со структурированными данными, такими как финансовые записи и системы инвентаризации.

Преимущество реляционных баз данных -- их строгость в отношении структуры данных. Все данные должны быть представлены в виде таблиц, что помогает избежать ошибок и дублирования данных. При этом реляционные базы данных обладают высокой производительностью при работе с небольшими и средними объемами информации.

Основные компоненты реляционных баз данных:

1) Таблицы -- основная единица хранения данных, где строки представляют отдельные записи, а столбцы — атрибуты этих записей;

2) Индексы используются для ускорения поиска данных в таблицах;

3) Связи позволяют объединять данные из разных таблиц на основе общих атрибутов, что позволяет создавать более сложные запросы;

4) Язык SQL -- программируемый язык, используемый для взаимодействия с базой данных, выполнения запросов на добавление, удаление, изменение и извлечение данных.

При рассмотрении системы управления реляционными базами данных организации могут выбирать между решениями с открытым исходным кодом и корпоративными платформами.

Типичные системы СУБД, поддерживающие реляционные базы данных, включают MySQL, PostgreSQL, Oracle и Microsoft SQL Server. Эти системы широко используются в коммерческих и государственных структурах, где важна целостность и консистентность данных.

\subsection{Нереляционные базы данных (NoSQL)}
Нереляционные базы данных представляют собой более гибкую альтернативу реляционным базам данных. Они предназначены для работы с неструктурированными или полуструктурированными данными, такими как текст, JSON, XML или графовые данные. В отличие от реляционных СУБД, где данные организованы в строгие таблицы, базы данных NoSQL позволяют хранить и обрабатывать данные более гибким способом. Системы NoSQL не требуют фиксированной схемы -- это означает, что разные записи могут иметь разные поля или структуры без необходимости обновления всей базы данных, подобно тому, как вы можете добавить новый столбец только в одну строку Excel, не изменяя все остальные строки.

Системы NoSQL предлагают различные модели данных: 

1) Документно-ориентированные базы данных (MongoDB, CouchDB) хранят данные в виде документов (чаще всего в формате JSON), что удобно для хранения данных с переменной структурой;

2) Колоночные базы данных (Apache Cassandra, HBase) -- данные хранятся в виде столбцов, а не строк, что позволяет эффективно обрабатывать большие объемы данных;

3) Графовые базы данных (Neo4j, ArangoDB) -- хранят информацию в виде узлов и рёбер, что позволяет эффективно моделировать и анализировать связи между объектами;

4) Ключ-значение базы данных (Redis, DynamoDB) -- хранят данные в виде пар «ключ-значение», что подходит для простых структур данных.

Одним из главных преимуществ баз данных NoSQL является их способность масштабироваться и работать с большими объемами данных, что делает их особенно полезными в условиях облачных вычислений и для обработки больших данных.

Основные отличия сводятся к следующему:

1) Реляционный использует фиксированные таблицы, нереляционный использует другие структуры и другие гибкие форматы;

2) Реляционные БД идеальны для структурированных данных, нереляционные -- для разнородных;

3) Реляционная обеспечивает немедленную согласованность, нереляционный подход может заменить некоторую согласованность на скорость и гибкость.
\subsection{Преимущества и недостатки реляционных и нереляционных баз данных}
Реляционные и нереляционные базы данных имеют свои сильные и слабые стороны, которые определяют их применимость в различных областях.

Преимущества реляционных баз данных: 

1) Строгая структура данных, что помогает поддерживать целостность информации;

2) Возможность использования мощных запросов с объединением таблиц и агрегированием данных;

3) Поддержка транзакций и обеспечение ACID (атомарности, консистентности, изолированности, долговечности), что важно для критически важных приложений. 

Недостатки реляционных баз данных:

1) Ограниченная масштабируемость при работе с очень большими объемами данных;

2) Высокая нагрузка на производительность при увеличении объема информации, особенно в распределенных системах;

3) Требования к жесткой структуре данных, что может быть неудобным при работе с неструктурированными данными.

Преимущества нереляционных баз данных: 

1) Высокая гибкость в работе с неструктурированными или полуструктурированными данными;

2) Хорошая масштабируемость и высокая производительность при обработке больших данных;

3) Легкость в интеграции с облачными вычислениями и распределенными системами.

Недостатки нереляционных баз данных:

1) Отсутствие строгой схемы данных, что может привести к ошибкам или несоответствиям;

2) Отсутствие поддержки ACID транзакций в некоторых системах;

3) Более сложные запросы и трудности с обработкой сложных взаимосвязанных данных.

Реляционные и нереляционные базы данных — это два типа хранения данных для бизнеса. Первый использует таблицы для организации информации, в то время как второй более разнообразен и может хранить данные в других типах структур, таких как графы или иерархии.

Реляционные и нереляционные базы данных удовлетворяют различные потребности в современном управлении данными. Каждый тип лучше подходит для конкретных сценариев, от платформ электронной коммерции до сложных систем аналитики.

Платформы электронной коммерции часто используют оба типа баз данных. Реляционные базы данных управляют заказами, платежами и запасами. Нереляционные базы данных обрабатывают каталоги продуктов и данные о поведении пользователей.

\section{Основные функции и возможности систем управления базами данных}
\section{Язык SQL как инструмент взаимодействия с базами данных}
Рост количества данных, необходимость их хранения и обработки привели к тому, что возникла потребность в создании стандартного языка баз данных, который мог бы функционировать в многочисленных компьютерных системах различных видов. Действительно, с его помощью пользователи могут манипулировать данными независимо от того, работают ли они на персональном компьютере, сетевой рабочей станции или универсальной ЭВМ. Одним из языков, появившихся в результате разработки реляционной модели данных, является язык SQL (Structured Query Language), который в настоящее время получил очень широкое распространение и фактически превратился в стандартный язык реляционных баз данных. Стандарт на язык SQL был выпущен Американским национальным институтом стандартов (ANSI) в 1986 г., а в 1987 г. Международная организация стандартов (ISO) приняла его в качестве международного. 
\subsection{Типы команд SQL} 
Реализация в SQL концепции операций, ориентированных на табличное представление данных, позволила создать компактный язык с небольшим набором предложений. Язык SQL может использоваться как для выполнения запросов к данным, так и для построения прикладных программ. Основные категории команд языка SQL предназначены для выполнения различных функций, включая построение объектов базы данных и манипулирование ими, начальную загрузку данных в таблицы, обновление и удаление существующей информации, выполнение запросов к базе данных, управление доступом к ней и ее общее администрирование. 

Основные категории команд языка SQL:

- DDL – язык определения данных; 

- DML -- язык манипулирования данными; 

- DQL – язык запросов; 

- DCL – язык управления данными; 

- Команды администрирования данных; 

- Команды управления транзакциями.
\subsection{Определение структур базы данных (DDL)} 
Язык определения данных (Data Definition Language, DDL) позволяет создавать и изменять структуру объектов базы данных, например, создавать и удалять таблицы. 

Основными командами языка DDL являются следующие:

- CREATE TABLE;

- ALTER TABLE;

- DROP TABLE;

- CREATE INDEX;

- ALTER INDEX;

- DROP INDEX. 
\subsection{Манипулирование данными (DML)} 
Язык манипулирования данными (Data Manipulation Language, DML) используется для манипулирования информацией внутри объектов реляционной базы данных посредством трех основных команд: 

- INSERT;

- UPDATE;

- DELETE. 
\subsection{Выборка данных (DQL)}
Язык запросов DQL наиболее известен пользователям реляционной базы данных, несмотря на то, что он включает всего одну команду SELECT. Эта команда вместе со своими многочисленными опциями и предложениями используется для формирования запросов к реляционной базе данных. 
\subsection{Язык управления данными (DCL -- Data Control Language)}
Команды управления данными позволяют управлять доступом к информации, находящейся внутри базы данных. Как правило, они используются для создания объектов, связанных с доступом к данным, а также служат для контроля над распределением привилегий между пользователями. Команды управления данными следующие: 

- GRANT;

- REVOKE. 
\subsection{Команды администрирования данных}
С помощью команд администрирования данных пользователь осуществляет контроль над выполняемыми действиями и анализирует операции базы данных; они также могут оказаться полезными при анализе производительности системы. Не следует путать администрирование данных с администрированием базы данных, которое представляет собой общее управление базой данных и подразумевает использование команд всех уровней. 
\subsection{Команды управления транзакциями}
Существуют следующие команды, позволяющие управлять транзакциями базы данных:

- COMMIT;

- ROLLBACK;

- SAVEPOINT;

- SET TRANSACTION.
\subsection{Преимущества языка SQL} 
Язык SQL является основой многих СУБД, т.к. отвечает за физическое структурирование и запись данных на диск, а также за чтение данных с диска, позволяет принимать SQL - запросы от других компонентов СУБД и пользовательских приложений. Таким образом, SQL – мощный инструмент, который обеспечивает пользователям, программам и вычислительным системам доступ к информации, содержащейся в реляционных базах данных. 

Основные достоинства языка SQL заключаются в следующем:

- Стандартность;

- Независимость от конкретных СУБД; 

- Возможность переноса с одной вычислительной системы на другую; 

- Реляционная основа языка; 

- Возможность создания интерактивных запросов; 

- Возможность программного доступа к БД; 

- Обеспечение различного представления данных; 

- Возможность динамического изменения и расширения структуры БД;

- Поддержка архитектуры клиент-сервер. 

Любой язык работы с базами данных должен предоставлять пользователю следующие возможности: 

- Создавать базы данных и таблицы с полным описанием их структуры; 

- Выполнять основные операции манипулирования данными, в частности, вставку, модификацию и удаление данных из таблиц; 

- Выполнять простые и сложные запросы, осуществляющие преобразование данных.
 
Кроме того, язык работы с базами данных должен решать все указанные выше задачи при минимальных усилиях со стороны пользователя, а структура и синтаксис его команд – достаточно просты и доступны для изучения. И наконец, он должен быть универсальным, т.е. отвечать некоторому признанному стандарту, что позволит использовать один и тот же синтаксис и структуру команд при переходе от одной СУБД к другой. Язык SQL удовлетворяет практически всем этим требованиям. 

\subsection{Синтаксис операторов, типы данных} 
Оператор SQL состоит из зарезервированных слов, а также из слов, определяемых пользователем. Зарезервированные слова являются постоянной частью языка SQL и имеют фиксированное значение. Их следует записывать в точности так, как это установлено, нельзя разбивать на части для переноса с одной строки на другую. Слова, определяемые пользователем, задаются им самим (в соответствии с синтаксическими правилами) и представляют собой идентификаторы или имена различных объектов базы данных. Слова в операторе размещаются также в соответствии с установленными синтаксическими правилами. 

Идентификаторы языка SQL предназначены для обозначения объектов в базе данных и являются именами таблиц, представлений, столбцов и других объектов базы данных. Символы, которые могут использоваться в создаваемых пользователем идентификаторах языка SQL, должны быть определены как набор символов. Стандарт SQL задает набор символов, который используется по умолчанию, – он включает строчные и прописные буквы латинского алфавита (A-Z, a-z), цифры (0-9) и символ подчеркивания (\verb|_|). На формат идентификатора накладываются следующие ограничения: 

- Идентификатор может иметь длину до 128 символов;

- Идентификатор должен начинаться с буквы;

- Идентификатор не может содержать пробелы. 

Большинство компонентов языка не чувствительны к регистру. Поскольку у языка SQL свободный формат, отдельные SQL-операторы и их последовательности будут иметь более читаемый вид при использовании отступов и выравнивания. 

Язык, в терминах которого дается описание языка SQL, называется метаязыком. Синтаксические определения обычно задают с помощью специальной металингвистической символики, называемой Бэкуса Науэра формулами (БНФ). Прописные буквы используются для записи зарезервированных слов и должны указываться в операторах точно так, как это будет показано. Строчные буквы употребляются для записи слов, определяемых пользователем.
 
Символьные типы данных SQL содержат буквы, цифры и специальные символы. CHAR или CHAR(n) -- символьные строки фиксированной длины. Длина строки определяется параметром n. CHAR без параметра соответствует CHAR(1). Для хранения таких данных всегда отводится n байт вне зависимости от реальной длины строки. VARCHAR(n) -- символьная строка переменной длины. Для хранения данных этого типа отводится число байт, соответствующее реальной длине строки.

Целые типы данных поддерживают только целые числа (дробные части и десятичные точки не допускаются). Над этими типами разрешается выполнять арифметические операции и применять к ним агрегирующие функции (определение максимального, минимального, среднего и суммарного значения столбца реляционной таблицы). 

INTEGER или INT -- целое, для хранения которого отводится, как правило, 4 байта. (Замечание: число байт, отводимое для хранения того или иного числового типа данных зависит от используемой СУБД и аппаратной платформы, здесь приводятся наиболее «типичные» значения). Интервал значений от - 2147483647 до + 2147483648.

SMALLINT -- короткое целое (2 байта), интервал значений от - 32767 до +32768 
 
Вещественные типы данных описывают числа с дробной частью:

- FLOAT и SMALLFLOAT - числа с плавающей точкой (для хранения отводится обычно 8 и 4 байта соответственно);

- DECIMAL(p) - тип данных аналогичный FLOAT с числом значащих цифр p;

- DECIMAL(p,n) - аналогично предыдущему, p -- общее количество десятичных цифр, n -- количество цифр после десятичной запятой. 
 
Денежные типы данных описывают денежные величины. Если ваша система такого типа данных не поддерживает, то используйте DECIMAL(p,n). 

MONEY -- все аналогично типу DECIMAL(p,n). Вводится только потому, что некоторые СУБД предусматривают для него специальные методы форматирования. 

Дата и время -- используются для хранения даты, времени и их комбинаций. Большинство СУБД умеет определять интервал между двумя датами, а также уменьшать или увеличивать дату на определенное количество времени:

- DATE -- тип данных для хранения даты;

- TIME -- тип данных для хранения времени;

- INTERVAL -- тип данных для хранения временного интервала;

- DATETIME -- тип данных для хранения моментов времени (год + месяц + день + часы + минуты + секунды + доли секунд). 

Двоичные типы данных позволяют хранить данные любого объема в двоичном коде (оцифрованные изображения, исполняемые файлы и т.д.). Определения этих типов наиболее сильно различаются от системы к системе, часто используются ключевые слова: 

- BINARY;

- BYTE;

- BLOB.

Последовательные типы данных используются для представления возрастающих числовых последовательностей. 

SERIAL -- тип данных на основе INTEGER, позволяющий сформировать уникальное значение (например, для первичного ключа). При добавлении записи СУБД автоматически присваивает полю данного типа значение, получаемое из возрастающей последовательности целых чисел. 

Для всех типов данных имеется общее значение NULL – «не определено». Это значение имеет каждый элемент столбца до тех пор, пока в него не будут введены данные. При создании таблицы можно явно указать СУБД могут ли элементы того или иного столбца иметь значения NULL (это не допустимо, например, для столбца, являющего первичным ключом). 
\subsection{Создание, модификация и удаление таблиц}
\begin{verbatim}
CREATE TABLE <имя_таблицы> 
(<имя_столбца> <тип_столбца> 
[NOT NULL] 
[UNIQUE | PRIMARY KEY] 
[REFERENCES <имя_ таблицы> (<имя_столбца>)] , ...)
\end{verbatim}

Пользователь обязан указать имя таблицы и список столбцов. Для каждого столбца обязательно указываются его имя и тип, а также опционально могут быть указаны параметры: 

- NOT NULL -- в этом случае элементы столбца всегда должны иметь определенное значение (не NULL);

- Один из взаимоисключающих параметров UNIQUE -- значение каждого элемента столбца должно быть уникальным или PRIMARY KEY -- столбец является первичным ключом;

- \verb|REFERNCES <имя_мастер_таблицы> [<имя_столбца>]| -- эта конструкция определяет, что данный столбец является внешним ключом и указывает на ключ какой мастер таблицы он ссылается.

Контроль за выполнением указанных условий осуществляет СУБД.

Как бы тщательно не планировалась структура таблицы, иногда возникает необходимость внести в нее некоторые изменения. Предположим, что в уже сформированную таблицу необходимо добавить столбец. Эту операцию можно выполнять различными путями. Например, можно удалить таблицу со старой структурой и создать вместо нее новую таблицу с нужной структурой. Недостатком этого метода является то, что необходимо будет куда-то скопировать имеющиеся в таблице данные и переписать их в новую таблицу после ее создания. 

Специальная команда ALTER TABLE предназначена для модификации структуры таблицы. С ее помощью можно изменять свойства существующих столбцов, удалять или добавлять в таблицу столбцы, а также управлять ограничением целостности, как на уровне столбца, так и на уровне таблицы, т.е. выполнять следующие функции: 

- Добавить в таблицу определение нового столбца;

- Удалить столбец из таблицы; 

- Изменить значение по умолчанию для какого-либо столбца;

- Добавить или удалить первичный ключ таблицы;

- Добавить или удалить внешний ключ таблицы;

- Добавить или удалить условие уникальности;

- Добавить или удалить условие на значение. 
 
Команда ALTER TABLE берет на себя все действия по копированию данных во временную таблицу, удалению старой таблицы и созданию вместо нее новой таблицы с нужной структурой и последующим переписыванием в нее данных. Назначение многих параметров и ключевых слов команды ALTER TABLE аналогично назначению соответствующих параметров и ключевых слов команды CREATE TABLE. 

Рассмотрим основные режимы использования команды ALTER TABLE: 

- Добавление столбца; 

- Удаление столбца; 

- Модификация столбца.

Добавление столбца: 
\begin{verbatim}
ALTER TABLE <имя_таблицы> ADD     
(<имя_столбца> <тип_столбца>     
[NOT NULL]     
[UNIQUE | PRIMARY KEY]    
[REFERENCES <имя_мастер_таблицы> (<имя_столбца>)]    
,...) 
\end{verbatim}

Модификация столбца:
\begin{verbatim}
ALTER TABLE <имя_таблицы>   
ALTER COLUMN(<имя_столбца> <тип_столбца>    
[NOT NULL]    
[UNIQUE | PRIMARY KEY]    
[REFERENCES <имя_мастер_таблицы> (<имя_столбца>)]    ,...) 
\end{verbatim}

Изменение столбца невозможно, если: 

- Столбец участвует в ограничениях PRIMARY KEY или FOREIGN KEY; 

- На столбец наложены ограничения целостности, например UNIQUE (исключение – столбцы, имеющие тип данных переменной длины;

- Со столбцом связано значение по умолчанию. 

Определяя для столбца новый тип данных, следует помнить о том, что старый тип данных должен конвертироваться в новый. 

Удаление столбца:
\begin{verbatim}
ALTER TABLE <имя_таблицы> DROP 
(<имя_столбца> 
,...) 
\end{verbatim}

Нельзя удалять столбцы с ограничением целостности CHECK, FOREIGN KEY, UNIQUE или PRIMARY KEY, а также столбцы, для которых определены значения по умолчанию. 
 
Удаление таблиц: 
\begin{verbatim}
DROP TABLE <имя_таблицы> 
\end{verbatim}

Невозможно удалить таблицу, если на нее ссылается другая таблица. 
\subsection{Операторы манипулирования данными}
К этой группе относятся операторы добавления, изменения и удаления записей. 

Добавление новой записи в таблицу:
\begin{verbatim}
INSERT INTO <имя_таблицы> 
[<имя_столбца>,<имя_столбца>,..)] 
VALUES (<значение>,<значение>,..); 
\end{verbatim}

Список столбцов в данной команде не является обязательным параметром. В этом случае должны быть указаны значения для всех полей таблицы в том порядке, как эти столбцы были перечислены в команде CREATE TABLE. 
 
Модификация записей:
\begin{verbatim}
UPDATE <имя_таблицы> SET <имя_столбца>=<значение>,... 
WHERE <условие>] 
\end{verbatim}

Если задано ключевое слово WHERE и условие, то команда UPDATE применяется только к тем записям, для которых оно выполняется. Если условие не задано, UPDATE применяется ко всем записям. 

В качестве условия используются логические выражения над константами и полями. В условиях допускаются: 

1) Операции сравнения: > , < , >= , <= , = , <> , != . В SQL эти операции могут применяться не только к числовым значениям, но и к строкам ( «<» означает раньше, а «>» позже в алфавитном порядке) и датам («<» раньше и «>» позже в хронологическом порядке);

2) Операции проверки поля на значение NULL -- IS NULL, IS NOT NULL;

3) Операции проверки на вхождение в диапазон -- BETWEEN и NOT BETWEEN;

4) Операции проверки на вхождение в список -- IN и NOT IN;

5) Операции проверки на вхождение подстроки -- LIKE и NOT LIKE.

Отдельные операции соединяются связями AND, OR, NOT и группируются с помощью скобок. 
 
Удаление записей:
\begin{verbatim}
DELETE FROM <имя_таблицы> 
[ WHERE <условие> ] 
\end{verbatim}

Удаляются все записи, удовлетворяющие указанному условию. Если ключевое слово WHERE и условие отсутствуют, из таблицы удаляются все записи. 
\subsection{Организация запросов на выборку данных при помощи языка SQL} 
Для извлечения записей из таблиц в SQL определен оператор SELECT. С помощью этой команды осуществляется не только операция реляционной алгебры «выборка» (горизонтальное подмножество), но и предварительное соединение двух и более таблиц. Это наиболее сложное и мощное средство SQL, полный синтаксис оператора SELECT имеет вид: 
\begin{verbatim}
SELECT [ALL | DISTINCT] <список_выбора> 
    FROM <имя_таблицы> 
    [ WHERE <условие> ] 
    [ GROUP BY <имя_столбца>,... ] 
    [ HAVING <условие> ] 
    [ORDER BY <имя_столбца> [ASC | DESC],... ] 
\end{verbatim}

Порядок предложений в операторе SELECT должен строго соблюдаться (например, GROUP BY должно всегда предшествовать ORDER BY), иначе это приведет к появлению ошибок. 

Этот оператор всегда начинается с ключевого слова SELECT. В конструкции <список\verb|_|выбора>  определяется столбец или столбцы, включаемые в результат. Он может состоять из имен одного или нескольких столбцов, или из одного символа «*» (звездочка), определяющего все столбцы. 

Элементы списка разделяются запятыми. 

В том случае, когда нас интересуют не все записи, а только те, которые удовлетворяют некому условию, это условие можно указать после ключевого слова WHERE. 

При выполнении оператора SELECT результирующее отношение может иметь несколько записей с одинаковыми значениями всех полей. Чтобы исключить повторяющиеся записи из выборки используется ключевое слово DISTINCT. Ключевое слово ALL указывает, что в результат необходимо включать все строки.  
 \subsection{Выборка из нескольких таблиц} 
Очень часто возникает ситуация, когда выборку данных надо производить из отношения, которое является результатом слияния двух других отношений. Для выполнения операции такого рода в операторе SELECT после ключевого слова FROM указывается список таблиц, по которым производится поиск данных. После ключевого слова WHERE указывается условие, по которому производится слияние. 

Следует обратить внимание на то, что когда в разных таблицах присутствуют одноименные поля, то для устранения неоднозначности перед именем поля указывается имя таблицы и знак «.» (точка). Хорошее правило: имя таблицы указывать всегда.

Замечание -- имеется возможность производить слияние и более чем двух таблиц. 
\subsection{Сортировка и группировка данных при помощи языка SQL}
Группировка данных в операторе SELECT осуществляется с помощью ключевого слова GROUP BY и ключевого слова HAVING, с помощью которого задаются условия разбиения записей на группы.
 
GROUP BY неразрывно связано с агрегирующими функциями, без них оно практически не используется. GROUP BY разделяет таблицу на группы, а агрегирующая функция вычисляет для каждой из них итоговое значение. Kлючевое слово HAVING работает следующим образом: сначала GROUP BY разбивает строки на группы, затем на полученные наборы накладываются условия HAVING. 
 
Для сортировки данных, получаемых при помощи оператора SELECT служит ключевое слово ORDER BY. С его помощью можно сортировать результаты по любому столбцу или выражению, указанному в <списке\verb|_|выбора>. Данные могут быть упорядочены как по возрастанию, так и по убыванию. Ключевое слово DESC задает обратный порядок сортировки, ключевое слов ASC (его можно опускать) - прямой порядок сортировки.  
\subsection{Функции в запросах SQL}
SQL позволяет выполнять различные арифметические операции над столбцами результирующего отношения. В конструкции <список\verb|_|выбора> можно использовать константы, функции и их комбинации с арифметическими операциями и скобками. 

В арифметических выражениях допускаются операции сложения (+), вычитания (-), деления (/), умножения (*), а также различные функции (COS, SIN, ABS -- абсолютное значение и т.д.). 
 
В SQL также определены так называемые агрегатные функции, которые совершают действия над совокупностью одинаковых полей в группе записей. Среди них: 

- AVG(<имя поля>) -- среднее по всем значениям данного поля;

- COUNT(<имя поля>) или COUNT (*) -- число записей;
 
- MAX(<имя поля>) -- максимальное из всех значений данного поля;
 
- MIN(<имя поля>) -- минимальное из всех значений данного поля;

- SUM(<имя поля>) -- сумма всех значений данного поля.

Следует учитывать, что каждая агрегирующая функция возвращает единственное значение. 

Область действия данных функции можно ограничить с помощью логического условия. 

Часто, текстовые значения заполняются пользователями программного обеспечения по-разному: кто пишет Ф.И.О. с заглавной буквы, кто нет; кто-то пишет все заглавными буквами. Многие отчетные формы требуют унифицированного подхода, да и не только отчетные формы. Для решения этой задачи в SQL есть две функции UCASE -- преобразует символы строки в верхний регистр и LCASE -- преобразует символы строки в нижний регистр. 

MID(<text>, <start\verb|_|num>, <num\verb|_|chars>) -- возвращает строку символов из середины текстовой строки с учетом начальной позиции и длины, где text -- текстовая строка, из которой нужно извлечь символы, или столбец, содержащий текст; start\verb|_|num -- положение первого символа, который необходимо извлечь (начинаются с 1); num\verb|_|chars -- число возвращаемых символов.

Иногда приходится в качестве аргументов функции MID использовать выражения с функцией LEN: 
 
LEN(column\verb|_|name) -- возвращает длину значения в поле записи. Функция LEN( ) исключает из подсчета конечные пробелы.

\section{Концепции целостности и надежности в системах хранения данных}
\section{Облачные подходы и онлайн-инструменты для работы с базами данных}
Современный этап развития технологий хранения данных характеризуется активным переходом к облачным решениям. Этот переход обусловлен необходимостью обработки растущих объемов информации при обеспечении масштабируемости и экономической эффективности. Российский рынок облачных услуг демонстрирует устойчивый рост, подтверждая актуальность внедрения облачных подходов.
\subsection{Основополагающие модели облачных услуг}
IaaS предоставляет базовые вычислительные ресурсы как услугу, включая виртуальные машины, системы хранения и сетевую инфраструктуру. Потребитель управляет операционными системами, приложениями и данными, сохраняя контроль над средой выполнения.

PaaS предлагает платформу для разработки и выполнения приложений. Провайдер управляет инфраструктурой и промежуточным программным обеспечением, а потребитель сосредотачивается на разработке приложений и управлении данными. Модель DBaaS относится к этой категории.

SaaS предоставляет готовое программное обеспечение как услугу через веб-интерфейс. Потребитель использует приложения провайдера без управления инфраструктурой и платформой.

XaaS представляет собой обобщающую концепцию, описывающую тенденцию перевода традиционных IT-услуг в облачную модель предоставления «как услуга».
\subsection{Классификация облачных моделей баз данных}
DBaaS представляет собой модель предоставления услуг, при которой провайдер полностью управляет инфраструктурой базы данных. Данная модель основана на принципе разделения ответственности: провайдер обеспечивает работу инфраструктуры, а пользователь работает с данными.

Преимущества:

- Снижение капитальных затрат на оборудование;

- Автоматическое масштабирование ресурсов;

- Встроенные механизмы резервного копирования;

- Регулярные обновления безопасности;

- Профессиональная техническая поддержка.

Yandex Cloud Managed Service for PostgreSQL предлагает полностью управляемые базы данных с автоматическим масштабированием. SberCloud. Базы данных обеспечивают высокую доступность и встроенные механизмы репликации.
\subsection{Бессерверные базы данных}
Бессерверные базы данных представляют архитектуру, где масштабирование происходит на уровне отдельных операций. Эта модель исключает необходимость управления серверами и емкостью хранилища.

Преимущества:

- Экономическая эффективность для переменных нагрузок;

- Автоматическое масштабирование от нуля;

- Оплата только за выполненные операции;

- Упрощение процесса развертывания.

Yandex Cloud Serverless Databases обеспечивают автоматическое масштабирование для приложений с нестабильным трафиком. Российские разработки в области бессерверных вычислений активно развиваются и внедряются в различных отраслях.
\subsection{Инструменты для работы с облачными базами данных}
Веб-консоли представляют графические интерфейсы для администрирования баз данных через браузер. Они обеспечивают централизованное управление всеми аспектами работы с данными.

Преимущества:

- Упрощение административных задач;

- Мониторинг производительности в реальном времени;

- Визуализация метрик использования;

- Централизованное управление доступом.

Панели управления Yandex Cloud и SberCloud предоставляют комплексные инструменты для работы с базами данных. Эти решения поддерживают русский язык и адаптированы под российских пользователей.
\subsection{Инструменты миграции данных}
Сервисы миграции обеспечивают перенос данных из локальных систем в облачную среду. Они поддерживают различные сценарии переноса с сохранением целостности данных.

Преимущества:

- Минимальное время простоя;

- Автоматическое преобразование форматов;

- Контроль целостности данных;

- Поддержка различных СУБД.

Отечественные решения для миграции данных учитывают требования российского законодательства. Эти инструменты обеспечивают соответствие 152-ФЗ и другим нормативным актам.
\subsection{Платформы бизнес-аналитики}
BI-платформы предоставляют инструменты для анализа и визуализации данных. Они позволяют создавать интерактивные отчеты и дашборды.

Преимущества:

- Интерактивная визуализация данных;

- SQL-интерфейсы для аналитических запросов;

- Совместная работа над отчетами;

- Интеграция с различными источниками.

Yandex DataLens предлагает мощные инструменты для бизнес-аналитики. Российские BI-платформы адаптированы под особенности местного рынка и поддерживают русскоязычный интерфейс.
\subsection{Безопасность и надежность}
Облачные провайдеры реализуют многоуровневую систему защиты, соответствующую российским и международным стандартам.

Преимущества:

- Шифрование данных на всех этапах;

- Управление доступом на основе ролей;

- Соответствие 152-ФЗ;

- Регулярные аудиты безопасности.

Российские облачные провайдеры обеспечивают полное соответствие требованиям законодательства. Локализация данных в пределах РФ гарантирует соблюдение нормативных требований.
\subsection{Обеспечение отказоустойчивости}
Системы построены на принципах географической репликации и автоматического восстановления.

Преимущества:

- Высокая доступность сервисов;

- Автоматическое переключение при сбоях;

- Резервное копирование;

- Мониторинг 24/7.

Отечественные облачные платформы развертываются в нескольких дата-центрах на территории России. Это обеспечивает отказоустойчивость и соблюдение требований к локализации данных.
\end{document}